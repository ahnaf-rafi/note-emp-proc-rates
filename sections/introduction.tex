%! TEX root = ../note-emp-proc-rates.tex

\section{Introduction}

Let \(n \in \mathbb{N}\), \((\Omega, \mathscr{A}, \Pr)\) be an underlying
probability space and \((\mathcal{X}, \mathscr{X})\) be a measurable space.
Let \(Q\) be a signed measure on \((\mathcal{X}, \mathscr{X})\),
and define the following:
\begin{equation*}
  \begin{split}
  Q g :=
  & \, \int g (x) Q (\mathrm{d} x) \quad \text{if } g \in \mathscr{L}_{1} (Q),
  \\
  \text{and} \quad \|Q\|_{\mathcal{F}} :=
  & \, \sup_{f \in \mathcal{F}} |Q f| \quad \text{if} \quad \mathcal{F} \subseteq
  \mathscr{L}_{1} (Q).
  \end{split}
  % \label{eqn--integral-functional-and-norm}
\end{equation*}
Let \(X_{1}, \dots, X_{n}\) be independent \(\mathcal{X}\)-valued random
elements defined on \((\Omega, \mathscr{A}, \Pr)\) and denote the \(\Pr\)-law of
\(X_{i}\) by \(P_{i}\).
Define the measures:
\begin{equation*}
  \begin{gathered}
    \mathbb{P} (A) \equiv \mathbb{P}_{n} (A) := \frac{1}{n} \sum_{i = 1}^{n}
    \delta_{\left\{ X_{i} \right\}} (A) = \frac{1}{n} \sum_{i = 1}^{n}
    \mathbf{1} \left\{ X_{i} \in A \right\} \\
    \text{and} \quad \overline{P} (A) \equiv \overline{P}_{n} (A) := \frac{1}{n}
    \sum_{i = 1}^{n} P_{i} (A).
  \end{gathered}
  % \label{eqn--emp-and-avg-prob-meas}
\end{equation*}
For \(\mathcal{F} \subseteq \mathscr{L}_{1} (\overline{P})\),
the empirical process is the map
\(f \mapsto \left( \mathbb{P} - \overline{P} \right) [f] := \frac{1}{n} \sum_{i
= 1}^{n} \left( f \left( X_{i} \right) - P_{i} f \right)\).
It will be useful for later results to define the supremum second moment of
\(\mathcal{F}\): for a probability measure \(Q\), define
\begin{equation}
  \kappa_{2} (Q, \mathcal{F}) := \sup_{f \in \mathcal{F}} \sqrt{Q \left[
  f^{2} \right]}.
  \label{eqn--F-2nd-moment-max}
\end{equation}
Finally, our bounds will all depend on a measure of complexity of the class
\(\mathcal{F}\).
As in the rest of the empirical process literature, we will mainly use
covering (and packing) numbers under \(\mathscr{L}_{p}\) norms.
\Cref{def--Lr-covering-number} provides a definition of this concept.

\begin{definition}[\(\mathscr{L}_{p} (Q)\)-covering numbers]
\label{def--Lr-covering-number}
Let \(Q\) be a measure on \(\left( \mathcal{X}, \mathscr{X} \right)\) and let
\(\mathscr{L}_{p} (Q)\) denote the space of \(Q\)-\(p\)-integrable functions for
\(p \in [1, \infty)\) and the space of \(Q\)-essentially bounded functions for
\(p = \infty\).
For \(\mathcal{F} \subseteq \mathscr{L}_{p} (Q)\) and \(\varepsilon > 0\),
define
\begin{equation*}
  N \left( \varepsilon, \mathscr{L}_{p} (Q), \mathcal{F} \right) := \min \left\{
  k \in \mathbb{N} : \exists f_{1}, \dots, f_{k} \in \mathcal{F} \text{ s.t. }
  \min_{1 \leq j \leq k} \left( Q \left[ \left| f - f_{j} \right|^{p} \right]
  \right)^{\frac{1}{p}} < \varepsilon \ \forall f \in \mathcal{F} \right\}.
\end{equation*}
\end{definition}

Our main goal here is to provide upper bounds for the tail outer probability
\(\Pr^{\ast} \left\{ \left\| \mathbb{P} - \overline{P} \right\|_{\mathcal{F}} >
y \right\}\) as a function of \(y\), \(n\), \(\mathcal{F}\) and
measures of complexity (i.e. covering numbers) and variability (i.e. suprema of
second moments and variances) of \(\mathcal{F}\).
This is considered for the case where functions in \(\mathcal{F}\) are
themselves uniformly bounded in magnitude by \(1\).
A second goal is then to explore how these probability bounds can then be used
to derive rates of convergence.
A tertiary aim will be to characterize how and when the chaining method offers
an improvement over more crude methods using only covering
numbers, for both probability inequalities and convergence rate results.

%%% Local Variables:
%%% mode: LaTeX
%%% TeX-master: "../note-emp-proc-rates"
%%% End:

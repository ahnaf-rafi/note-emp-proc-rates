%! TEX root = ../note-emp-proc-rates.tex

\section{Introduction}

Let \(n \in \mathbb{N}\), \((\Omega, \mathscr{A}, \Pr)\) be an underlying
probability space and \((\mathcal{X}, \mathscr{X})\) be a measurable space.
Let \(X_{1}, \dots, X_{n}\) be independent \(\mathcal{X}\)-valued random
elements defined on \((\Omega, \mathscr{A}, \Pr)\) and denote the \(\Pr\)-law of
\(X_{i}\) by \(P_{i}\).
Define the measures:
\begin{equation*}
  \mathbb{P} (A) \equiv \mathbb{P}_{n} (A) := \frac{1}{n} \sum_{i = 1}^{n}
  \delta_{\left\{ X_{i} \right\}} (A) = \frac{1}{n} \sum_{i = 1}^{n} \mathbf{1}
  \left\{ X_{i} \in A \right\} \quad \text{and} \quad \overline{P} (A) \equiv
  \overline{P}_{n} (A) := \frac{1}{n} \sum_{i = 1}^{n} P_{i} (A).
\end{equation*}
For an arbitrary (signed) measure \(Q\) on \((\mathcal{X}, \mathscr{X})\), and
\(f \in \mathscr{L}_{1} (Q)\), let \(Q f = \int f (x) \; \mathrm{d} x\).
Furthermore, let \(\mathcal{F}\) be a class of \(\mathscr{X}\)-measurable
real-valued functions.
The empirical process is
\begin{equation*}
  f \mapsto \left( \mathbb{P} - \overline{P} \right) f := \frac{1}{n}
  \sum_{i = 1}^{n} \left( f \left( X_{i} \right) - P_{i} f \right).
\end{equation*}
Its sup-norm as a real-valued stochastic process indexed by \(\mathcal{F}\)
is denoted
\begin{equation*}
  \left\| \mathbb{P} - \overline{P} \right\|_{\mathcal{F}} = \sup_{f \in
  \mathcal{F}} \left| \left( \mathbb{P} - \overline{P} \right) f \right|.
\end{equation*}

Our first goal here is to provide upper bounds for
\begin{equation*}
  \Pr \left\{ \left\| \mathbb{P} - \overline{P} \right\|_{\mathcal{F}} > y
  \right\}
\end{equation*}
as a function of \(y\), \(n\), the class \(\mathcal{F}\) and measures of
complexity (i.e. covering numbers) and variability (i.e. second moment and
variance suprema) of \(\mathcal{F}\).
This is considered for the case where functions in \(\mathcal{F}\) are
themselves uniformly bounded in magnitude by \(1\).
The second goal is then to explore how these probability bounds can then be used
to derive rates of convergence.
A tertiary consideration will be to characterize how and when the chaining
method offers an improvement over more crude methods using only covering
numbers, for both probability inequalities and convergence rate results.

%%% Local Variables:
%%% mode: LaTeX
%%% TeX-master: "../note-emp-proc-rates"
%%% End:

% LocalWords:  cdf ecdf

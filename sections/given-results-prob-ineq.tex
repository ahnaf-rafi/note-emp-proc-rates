%! TEX root = ../note-emp-proc-rates.tex

\section{Useful preliminary probability inequalities}

Here, it will be useful to define the symmetrized empirical process:
\begin{equation*}
  \mathbb{P}^{\circ} f \equiv \mathbb{P}^{\circ}_{n} f := \frac{1}{n} \sum_{i =
  1}^{n} \varepsilon_{i} f \left( X_{i} \right),
\end{equation*}
where \(\varepsilon_{1}, \dots, \varepsilon_{n} \overset{\mathrm{iid}}{\sim}
\text{Rademacher}\) with \(\left\{ \varepsilon_{i} \right\}_{i = 1}^{n} \indep
\left\{ X_{i} \right\}_{i = 1}^{n}\).

The following is a useful result from
\citet{2023vandervaartWeakConvergenceEmpirical}, where it is stated as Lemma
2.3.7.
In the original statement, the final stipulation is made for iid processes, but
the adjustment made here is minor and clearly holds true as well by a simple
application of Chebyshev's inequality.

\begin{lemma}[\citet{2023vandervaartWeakConvergenceEmpirical}, Lemma 2.3.7,
p. 176]
\label{lem--vdvw23-lem-2-3-7}
Let \(Z_{1}, \dots, Z_{n}\) be arbitrary independent \(\mathcal{F}\)-indexed
stochastic processes, \(\mu_{1}, \dots, \mu_{n} : \mathcal{F} \to \mathbb{R}\)
be arbitrary functions, and \(\varepsilon_{1}, \dots, \varepsilon_{n}
\overset{\mathrm{iid}}{\sim}
\text{Rademacher}\) with \(\left\{ \varepsilon_{i} \right\}_{i = 1}^{n} \indep
\left\{ Z_{i} \right\}_{i = 1}^{n}\).
Then
\begin{equation*}
  \beta_{n} (y) \Pr^{\ast} \left( \left\| \sum_{i = 1}^{n} Z_{i}
  \right\|_{\mathcal{F}} > y \right) \leq 2 \Pr^{\ast} \left( 4 \left\| \sum_{i
  = 1}^{n} \varepsilon_{i} \left( Z_{i} - \mu_{i} \right) \right\|_{\mathcal{F}}
  > y \right),
\end{equation*}
for every \(y > 0\) and \(\beta_{n} (y) \leq \inf_{f \in \mathcal{F}} \Pr \left(
\left| \sum_{i = 1}^{n} Z_{i} (f) \right| < y / 2 \right)\).
In particular this is true for independent mean-zero finite variance processes,
and \(\beta_{n} (y) = 1 - \frac{4}{y^{2}} \sup_{f \in \mathcal{F}} \sum_{i =
1}^{n} \Var \left[ Z_{i} (f) \right]\).
\end{lemma}

Later on, we will combine \Cref{lem--vdvw23-lem-2-3-7} with some other
inequalities that can be found in proofs of results in
\citet{1984pollardConvergenceStochasticProcesses}.

%%% Local Variables:
%%% mode: LaTeX
%%% TeX-master: "../note-emp-proc-rates"
%%% End:

% LocalWords:  cdf ecdf
